\documentclass[DM,authoryear,toc]{lsstdoc}
% lsstdoc documentation: https://lsst-texmf.lsst.io/lsstdoc.html
\input{meta}

% Package imports go here.

% Local commands go here.

%If you want glossaries
%\input{aglossary.tex}
%\makeglossaries

\title{Statement of public data}

% Optional subtitle
% \setDocSubtitle{A subtitle}

\author{%
William O'Mullane
}

\setDocRef{DMTN-144}
\setDocUpstreamLocation{\url{https://github.com/lsst-dm/dmtn-144}}

\date{\vcsDate}

% Optional: name of the document's curator
% \setDocCurator{The Curator of this Document}

\setDocAbstract{%
DMLT were asked for a brief statement on public data releases.
}

% Change history defined here.
% Order: oldest first.
% Fields: VERSION, DATE, DESCRIPTION, OWNER NAME.
% See LPM-51 for version number policy.

\begin{document}

% Frequently for a technote we do not want a title page  uncomment this to remove the title page and changelog.
\mkshorttitle


\section{Statement}

The Rubin Observatory Construction Project has no remit to provide services or systems capable of supporting access by non-data-rights holders to data products after the two year proprietary period has elapsed.
That said, we do not regard doing so as an insurmountable task.
In particular:

\begin{itemize}

  \item{Data access services being developed during the construction era are designed to be capable of meeting any plausible public demand for access given appropriate hardware and network resources;}
  \item{Major commercial cloud vendors have a track record of hosting public datasets for free, and have expressed interest and willingness to engage with the Project on this topic;}
  \item{In-kind contributions may include capacity for making data available to the public;}
  \item{An ``object-lite'' catalog, containing only the most generally valuable columns provided by the full Object table, is sufficiently compact to be delivered via a content delivery network (as was done for the Gaia catalog).}

\end{itemize}

It is worth noting that this matter will not arise until the end of the third year of the survey; planning for deploying the above resources is therefore a question for the Operations Team.
Further discussions on various ways of serving parts or all of the data are discussed in \citeds{LPM-251}.


\appendix
% Include all the relevant bib files.
% https://lsst-texmf.lsst.io/lsstdoc.html#bibliographies
\section{References} \label{sec:bib}
\bibliography{local,lsst,lsst-dm,refs_ads,refs,books}

% Make sure lsst-texmf/bin/generateAcronyms.py is in your path
%\section{Acronyms} \label{sec:acronyms}
%\input{acronyms.tex}
% If you want glossary uncomment below -- comment out the two lines above
%\printglossaries





\end{document}
