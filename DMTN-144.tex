\documentclass[DM,authoryear,toc]{lsstdoc}
% lsstdoc documentation: https://lsst-texmf.lsst.io/lsstdoc.html
\input{meta}

% Package imports go here.

% Local commands go here.

%If you want glossaries
%\input{aglossary.tex}
%\makeglossaries

\title{Distribution of Rubin Observatory data outside the data rights community}

% Optional subtitle
% \setDocSubtitle{A subtitle}

\author{%
William O'Mullane
}

\setDocRef{DMTN-144}
\setDocUpstreamLocation{\url{https://github.com/lsst-dm/dmtn-144}}

\date{\vcsDate}

% Optional: name of the document's curator
% \setDocCurator{The Curator of this Document}

\setDocAbstract{%
DMLT were asked for a brief statement on public data releases.
}

% Change history defined here.
% Order: oldest first.
% Fields: VERSION, DATE, DESCRIPTION, OWNER NAME.
% See LPM-51 for version number policy.

\begin{document}

% Frequently for a technote we do not want a title page  uncomment this to remove the title page and changelog.
\mkshorttitle


\section{Statement}

The Rubin Observatory construction project has no remit to provide public access to data after data rights have lapsed.
The first instance of this would be around Data Release 3 and puts it firmly in the realm of operations. That said we do not see this an in insurmountable task.
Tools have been provided which are easily deployable and can scale to public access given resources.  The major cloud vendors host public datasets for free and have expressed interest/willingness.
There is interest in in-kind contributions to make the public data  accessible to the public.
An object-lite type catalog is sufficiently compact to consider delivery of files via a content delivery network (as was done for the Gaia catalog). Further discussions on various ways of serving parts or all of the data are discussed in \citeds{LPM-251}.



\appendix
% Include all the relevant bib files.
% https://lsst-texmf.lsst.io/lsstdoc.html#bibliographies
\label{sec:bib}
\bibliography{local,lsst,lsst-dm,refs_ads,refs,books}

% Make sure lsst-texmf/bin/generateAcronyms.py is in your path
%\section{Acronyms} \label{sec:acronyms}
%\input{acronyms.tex}
% If you want glossary uncomment below -- comment out the two lines above
%\printglossaries





\end{document}
